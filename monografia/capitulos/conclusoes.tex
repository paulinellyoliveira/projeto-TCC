Neste trabalho foi realizado um comparativo de desempenho entre um computador \textit{desktop} e uma Raspberry Pi. Para a realização dos testes, um AG para resolução do PCV foi utilizado de forma sequencial e paralela (uso da biblioteca OpenMP). Foram realizados dois testes, sendo que o primeiro era a execução do AG com entradas de tamanhos diferentes e o segundo execuções repetidas (total de 50) com tamanho de entrada fixo (30 cidades) em ambas as arquiteturas propostas. Com o tempo de execução obtido, foram aplicadas as métricas como $Sd$ e $Ef$, entre outras, para análise do desempenho e comportamento das arquiteturas.

Os resultados mostram um $Sd$ de 3 a 4,8 do \textit{notebook} em relação a Raspberry Pi no AG sequencial e um $Sd$ em torno de 3,3 do \textit{notebook} em relação a Raspberry Pi no AG paralelo. Estes resultados são esperados visto que a placa possui recursos mais limitados em relação ao \textit{notebook}, dos quais podemos destacar:

\begin{enumerate}
	\item Memória secundária: o armazenamento da Raspberry Pi é um cartão MicroSD (memória \textit{flash}) que é mais lenta se comparada a do \textit{notebook} (HD SATA).
	
	\item Memória primária: a memória da Raspberry Pi (DDR2 900 MHz) possui frequência menor que a do \textit{notebook} (DDR3 1,33GHz).
	
	\item Arquitetura x86 (híbrida de RISC e CISC) do \textit{notebook} possui instruções complexas já implementadas, enquanto que a arquitetura ARM (puramente RISC) não tem, o que aumenta a quantidade de instruções necessárias para executar determinadas tarfeas, se comparada com a x86.
\end{enumerate}

Por outro lado, a placa Raspberry Pi apresentou melhor eficiência na utilização de seus núcleos quando comparada com o \textit{notebook}. Isso se deve principalmente a otimização e robustez presentes da placa devido ao fato de ter seus recursos mais limitados, o que inclui um SO mais enchuto.

Por fim, pode-se concluir que a placa Raspberry Pi possui desempenho significativo que, aliado a outras vantagens (baixo custo, estabilidade, recursos como porta \textit{ethernet}, wi-fi, GPIO, entre outros), torna a placa uma alternativa barata e eficiente para aplicações de baixo custo e que não necessitam de tempo de resposta como fator crítico.

Pode-se destacar como principais limitações a falta de um ambiente de rede (\textit{cluster}) para testes com aplicações distribuídas, falta de testes com outros tipos de aplicações que exijam carga de processamento e análises utilizando softwares do tipo \textit{benchmark}. O consumo de energia e o desempenho com e sem um dissipador de calor para a Raspberry Pi também não foram analisados.

Como sugestão para trabalhos futuros:

\begin{enumerate}
	\item Testes com aplicações distribuídas, como por exemplo, utilizando \textit{cluster} homogêneo.
	
	\item Testes com aplicações distribuídas, como por exemplo, utilizando \textit{cluster} heterogêneo.
	
	\item Comparação com dispositivos similares a Raspberry Pi, por exemplo, a Orange Pi e a BeagleBone, entre outras.
	
	\item Testes com outras classes de algoritmos, por exemplo, para processamento gráfico, processamento de digital de imagens (PDI), mineração de dados em grandes volumes de informações, entre outras.
	
	\item Análise do consumo de energia, calor dissipado e a diferença de desempenho com e sem um dissipador de calor para a placa.
\end{enumerate}